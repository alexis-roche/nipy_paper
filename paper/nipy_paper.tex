\documentclass{bioinfo}
\usepackage{url}

\usepackage[british,english]{babel}
\usepackage{mathpazo}
\usepackage[T1]{fontenc}
% \usepackage[latin9]{inputenc}
\usepackage{float}
\usepackage{amsmath}
\usepackage{graphicx}
\usepackage{setspace}
\usepackage{amssymb}
\usepackage{natbib}
\usepackage[title]{appendix}
\usepackage{siunitx}
\usepackage{chngcntr}
\usepackage{algorithmic}
\renewcommand{\algorithmicrequire}{\textbf{Input:}}
\renewcommand{\algorithmicensure}{\textbf{Output:}}

\usepackage{multirow}
\usepackage{rotating}

\makeatletter
\newfloat{algorithm}{H}{loa}[section]
\floatname{algorithm}{Algorithm}
\counterwithout{algorithm}{algorithm}
\def\argmin{\mathop{\operator@font arg\,min}} 
\def\argmax{\mathop{\operator@font arg\,max}} 
\makeatother

\copyrightyear{}
\pubyear{}

\begin{document}
\firstpage{1}

\title[nipy-paper]{Analysis of functional imaging data with NIPY}

% Alexis Roche1, Bertrand Thirion2, Gael Varoquaux2, Jonathan Taylor3, Matthew Brett4 and nipy contributors5
% 
% 1. University Hospital Lausanne/Siemens Medical Solutions, CIBM, Lausanne, Switzerland
% 2. Neurospin, CEA Saclay, Gif sur Yvette, France
% 3. Stanford University, Stanford, CA, USA,
% 4. University of California, Berkeley, CA, USA,
% 5. https://github.com/nipy/nipy/contributors

\author[Garyfallidis, Brett, Amirbekian, Rokem,  van der Walt, Descoteaux, 
Nimmo-Smith]{Eleftherios~Garyfallidis\,$^{1,2,*}$, Matthew~Brett\,$^{3}$,
  Bago~Amirbekian\,$^{4}$, Ariel~Rokem\,$^{5}$, Stefan~van der Walt\,$^{7}$, Maxime~Descoteaux\,$^{2}$, Ian~Nimmo-Smith\,$^{6}$ \footnote{to whom correspondence should be addressed. e-mail: garyfallidis@gmail.com}}

\address{\,$^{1}$University of Cambridge, Cambridge, UK\\
  \,$^{2}$University of Sherbrooke, Sherbrooke, CA\\ 
  \,$^{3}$University of California, Henry H. Wheeler, Jr. Brain Imaging Center, Berkeley, CA.\\
  \,$^{4}$University of California, San Francisco, CA, USA\\    
  \,$^{5}$Stanford University, Stanford, CA, USA\\
  \,$^{6}$MRC Cognition and Brain Sciences Unit, Cambridge, UK\\
  \,$^{7}$Stellenbosch University, Stellenbosch, South Africa\\  
  }


\history{}

\editor{}

\maketitle

\begin{abstract}
\noindent

NIPY is a library and application for analysing functional imaging data.   We
intended the project to be a project open to anyone who wanted to contribute
high-quality code.  Our aim is to make it easier for researchers to understand
the analysis and build their own tools, by writing code that is clear,
well-adapted to the scientific problem, and fast enough to run in reasonable
time.

Using Python has made it easier for us to work together because of the
readability of the language, the popularity of the language among open-source
developers, and Python’s emphasis on testing and documentation.

The package developed from a collaboration between researchers in Stanford,
Berkeley and CEA Neurospin in France.

Notable features of the NIPY package are:

Scriptable image diagnostics including flexible PCA Combined slice-timing and
movement correction for functional MRI images Flexible registration model with
pluggable cost functions and optimization algorithms Common spatial models of
function encountered in neuroimaging  (cluster-level models, parcellations )
Fast and flexible specification of regression models within and across
subjects and groups Visualization of statistical brain maps in 2D and 3D

We will describe these features of the package with examples of their use.

\section{Keywords:} Python, Diffusion MRI, Diffusion Tensor model,
Deconvolution, Medical imaging, Open source software, Deterministic
tractography, Probabilistic tractography, Visualisation.

\end{abstract}

\section{Skeleton}

Here is a possible outline for the paper:

\begin{verbatim}
Philosophy/Mission
General Design Aspects
File Formats 
Preprocessing
   Eddy currents?
   Denoising?
Reconstruction
   DTI 
   DSI
   QBall
   Spherical Deconvolution 
Tracking
   Deterministic
   Probabilistic
Post-tracking
   Segmentation
   Track_counts
   Track Lengths & other statistics
   Connectivity matrix?
Conclusions
\end{verbatim}

\section{Philosophy and Mission}

Someone will write this.

\section{LaTex Formatting Stuff}

This is to show how graphics (EPS) files are included. We use EPS for
speed. The first one is spread across both columns, and the second one
is just in a single column:

\begin{figure*}
\centerline{\includegraphics[width=160mm]{Figures/Fig_4_cst_simplification_relabeled_triple.eps}}
\caption{This is the figure caption - and a label to refer to it in the text \label{Fig:big_picture}}
\end{figure*}

When we want to refer to this figure we use the label (see
Fig.~\ref{Fig:big_picture}).

\begin{figure}
\includegraphics[scale=0.15]{Figures/Fig_11_MDF_arcuate}
\centering{}
\caption{Color coding shows MDF distances from QB centroid to every
  other track in the bundle.\label{Fig:little_picture}}
\end{figure}

Here are some displayed equations (see Eq.~\ref{eq:direct_flip_distance}):
\begin{eqnarray}
  d_{\textrm{direct}}(s, t) = d(s, t) & = & \frac{1}{K}\sum_{i=1}^{K}|s_{i}-t_{i}|,\nonumber\\
  d_{\textrm{flipped}}(s, t) & = & d(s,t^F) = d(s^F,t),\nonumber\\
  \textrm{MDF}(s, t) & = & \min(d_{\textrm{direct}}(s, t), d_{\textrm{flipped}}(s, t))\label{eq:direct_flip_distance}.
\end{eqnarray}

Inline mathematics goes like this: $\frac{1}{K}\sum_{i=1}^{K}|s_{i}-t_{i}|$

Here we have an example of a table (see Table~\ref{Table_1}).

\begin{table}[th] \processtable{QB centroids performance compared with
random subsets\label{Table_1}} {\begin{tabular}{rrrr} %\hline Thresholds &
Comparison & Coverage \% (s.d.) & Overlap (s.d.) \\ \hline
\multirow{2}{*}{$10$~mm/$10$~mm} & QB Centroids & 99.96 (0.007) & 2.44
(0.08)\\ & Random & 90.49 (0.41) & 6.16 (0.55)\\ \hline
\multirow{2}{*}{$20$~mm/$20$~mm} & QB Centroids & 99.99 (0.004) & 3.54
(0.18)\\ & Random & 95.86 (0.62) & 6.81 (0.93)\\ \hline
\end{tabular}}{}
\end{table}

References go like this: in parentheses
\citep{Garyfallidis_thesis,Mori1999}, and in running text
\citet{Garyfallidis_thesis}.

\section*{Acknowledgments}
Who do we need to acknowledge?

\section*{Disclosure/Conflict-of-Interest Statement}
Is it true that there are no conflicts of interest relating to this
work?

\selectlanguage{british}%
\bibliographystyle{apalike2}
%\bibliographystyle{plainnat}
%\bibliographystyle{IEEEabrv, IEEEtran}
%\bibliographystyle{IEEEtran}
%\bibliographystyle{elsarticle-harv}
\selectlanguage{english}
\bibliography{scilBibTex}

\end{document}
